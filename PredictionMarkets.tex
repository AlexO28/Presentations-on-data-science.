\documentclass{beamer}
\usepackage{graphicx}
\usetheme{Berlin}
\title{Prediction Markets.}
\author{Alexey Osipov}
\date{20.02.2020}
\begin{document}
\maketitle
\begin{frame}[fragile]
\frametitle{Equal probabilities experiment.}
\begin{itemize}
	\item Suppose we vote on an unbiased coin (50\%-50\%).

What should be the ratio of votes?

\ 

\ 

\ 

\item Suppose we bet instead of voting?

What should be the ratio of volumes?
\end{itemize}
\end{frame}
\begin{frame}[fragile]
\frametitle{Inequal probabilities experiment.}
\begin{itemize}
	\item Suppose we vote on a biased coin (80\%-20\%).

What should be the ratio of votes?

\ 

\ 

\ 

\item Suppose we bet instead of voting?

What should be the ratio of volumes?
\end{itemize}
\end{frame}
\begin{frame}[fragile]
\frametitle{What is a prediction market?}

It is a market of betting on events with a random outcome.
The probabilities of the outcome are identified from the bets.

In most cases the outcome probability is almost constant.

\ 

Examples of prediction markets:
\begin{enumerate}
	\item www.betfair.com (betting)
	\item www.crowdmed.com (medicine)
\end{enumerate}

Non-examples:
\begin{enumerate}
	\item stock market
	\item betting company
\end{enumerate}

? www.ladbrokes.com ?
\ 

Could they be used in insurance?
\end{frame}
\begin{frame}[fragile]
\frametitle{Bet for and bet against.}
Suppose we have an outcome.

\begin{itemize}
	\item A bet for the outcome by price $k$ is a contract: either get $k-1$ or loose $1$ depending on the event outcome.
  \item A bet against the outcome by price $k$ is a contract: either loose $k-1$ or get $1$ depending on the event outcome.
\end{itemize}
Naturally, they are complementary.

A bet against the outcome can be considered as the bet for the alternative outcome and vice versa (see the next slide).
\end{frame}
\begin{frame}[fragile]
\frametitle{Simple betting example.}

Let $k_1, k_2$ be the prices:
$$p_1=\frac{1}{k_1},\quad p_2=\frac{1}{k_2}, \qquad\frac{1}{k_1} + \frac{1}{k_2} = 1.$$
Let $s_1, s_2$ be the volumes. We expect:
$$s_1/p_1 = s_2/p_2$$
if the prices agree with probabilities.

The prices that agree with probabilities are \textbf{equilibrium prices}.

Suppose we are allowed to adjust the prices depending on the volumes.
If we do it in a reasonable way, the prices will converge to \textbf{equilibrium prices}.

The question is how soon...
\end{frame}
\begin{frame}[fragile]
\frametitle{Components of prediction market.}
\begin{itemize}
	\item The way of matching orders of different people into bets.
	\item The way of identifying the outcome probability by orders and bets.
	\item The policy about cancellation of orders.
	\item The policy about visibility of orders and bets.
	\item The market-maker (usually, it is used).
\end{itemize}
\end{frame}
\begin{frame}[fragile]
\frametitle{Order book.}
\begin{figure}
\centering
\includegraphics[width=50mm,scale=0.5]{OrderBookExample.jpg}
\end{figure}
\begin{figure}
\centering
\includegraphics[width=50mm,scale=0.5]{MatchingOrders.jpg}
\end{figure}
\end{frame}
\begin{frame}[fragile]
\frametitle{Identifying the probability.}
If the outcome probability does not change, at some point the prices should stabilize.

Once they are more or less stable, we can say
$$p = 1/k.$$

There are problems with this approach:
\begin{enumerate}
	\item What to do if there are not enough orders/bets 
	
	(\textbf{cold start})?
	\item What to do if bid-ask spread is too large 
	
	(\textbf{non-liquid market})?
\end{enumerate}

We try to resolve this problems with a model.
\end{frame}
\begin{frame}[fragile]
\frametitle{Key model components.}
\begin{enumerate}
  \item Small world assumption: $M$ is the maximal amount of capital.
	\item Utility functions:
	$$U(v, \lambda) = \frac{1-e^{-\lambda x/W}}{1-e^{-\lambda M/W}}.$$
	\item Predictions of gamblers come from the same distribution
	\
	
	(like $N(\mu, \sigma)$).
	\item Maximization of a likelihood function.
	\item Iterative optimization for finding $\mu$, $\sigma$, $\lambda$.
	\item Once $\mu$ (and $\lambda$, $\sigma$ are found) we may define $p$ and $k$ by:
	$$p=\mu,\qquad k=1/p.$$
\end{enumerate}
\end{frame}
\end{document}