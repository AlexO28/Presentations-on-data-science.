\documentclass{beamer}
\usepackage{listings}
\usetheme{Berlin}
\title{Collective risk model with \textbf{actuar}.}
\author{Alexey Osipov}
\date{09.06.2020}
\begin{document}
\maketitle
\begin{frame}[fragile]
\frametitle{Statement of the problem.}
\begin{itemize}
\item $N$ wildfires per year. Damage from a wildfire is $X$. What is aggregate annual damage from wildfires?
$$Y=X_1 + \ldots + X_N,\quad \textrm{ where }N\textrm{ is random.}$$
\item $N$ is frequency (Poisson).
$X$ is severity (Pareto).
Result is Compound Poisson distribution or collective risk model.
\item We want to find:
$$mean(Y)=?,\quad var(Y)=?,$$
$$VaR(Y, 0.99) = ?,\quad TVaR(Y, 0.99) = ?.$$
\end{itemize}
\end{frame}
\begin{frame}[fragile]
\frametitle{Pause 1.}
\begin{center}
Let us look into the code!
\end{center}
\end{frame}
\begin{frame}[fragile]
\frametitle{Methods from \textbf{actuar::aggregateDist}.}
\begin{enumerate}
	\item Monte Carlo.
	\item FFT.
	\item Panjer recursion.
	\item Normal approximation based methods.
\end{enumerate}

\textbf{actuar::aggregateDist} returns us \textbf{ecdf}.
\end{frame}
\begin{frame}[fragile]
\frametitle{Monte Carlo methods: description.}
\begin{itemize}
	\item Fix large number, like $K = 1000000$.
	\item Generate $K$ random years. For each year generate random number of wildfires. For each wildfire generate damage.
	\item Complexity is: $O(K\lambda)$.
\end{itemize}
\end{frame}
\begin{frame}[fragile]
\frametitle{Pause 2.}
\begin{center}
Let us look into the code!
\end{center}
\end{frame}
\begin{frame}[fragile]
\frametitle{Monte Carlo methods: critics.}
\begin{itemize}
	\item How to choose $K$?
	\item Should $K$ be connected with mean frequency?
	\item The method is \textbf{slow}, but \textbf{flexible}.
	\item What can be done further? Stratified sampling, Sobol sequence, Latin hypercube, Iman Conover, $\ldots$
\end{itemize}
\end{frame}
\begin{frame}[fragile]
\frametitle{FFT.}
\begin{itemize}
\item Discretize frequency.
\item Fix $n$, calculate $p_n$, probability to have exactly $n$ wildfires.
\item $Y_n = X_1 + \ldots + X_n$ is a convolution.
It is calculated via FFT.
$$FFT(Y_n) = FFT(X_1)\cdot\ldots\cdot FFT(X_n)$$
$$Y_n = FFT^{-1}(FFT(X_1)\cdot\ldots\cdot FFT(X_n))$$
\item complexity is $O(N\ln N)$ (for direct convolution calculation it is $O(N^3)$).
\end{itemize}
\end{frame}
\begin{frame}[fragile]
\frametitle{Pause 3.}
\begin{center}
Let us look into the code!
\end{center}
\end{frame}
\begin{frame}[fragile]
\frametitle{FFT, critics.}
\begin{itemize}
\item How to choose discretization step?
\item How to choose ranges?
\item Is uniform discretization obligatory?
\item Is it really FFT in actuar?
\end{itemize}
\end{frame}
\begin{frame}[fragile]
\frametitle{Panjer recursion.}
\begin{itemize}
\item frequency from Panjer class.
 $$P(N = k) = (a+b/k)P(N = k-1).$$
\item discretize severity with step $h$
\item calculate recursively $P(S = hk)$
\item frequency reduction trick: reduce frequency $m$ times by calculating $\ln(m)/\ln(2)$ of convolutions.
\item Complexity is $O(N^2)$.
\end{itemize}
\end{frame}
\begin{frame}[fragile]
\frametitle{Pause 4.}
\begin{center}
Let us look into the code!
\end{center}
\end{frame}
\begin{frame}[fragile]
\frametitle{Panjer recursion, critics.}
\begin{itemize}
\item How to choose discretization range?
\item How to choose discretization step?
\item Should we do frequency reduction trick?
\item Is uniform discretization obligatory?
\end{itemize}
\end{frame}
\begin{frame}[fragile]
\frametitle{Normal approximation.}
Method 1. 
\begin{itemize}
\item Calculate 2 first moments.
\item Approximate by normal distribution.
\end{itemize}

Method 2.
\begin{itemize}
\item Calculate 3 first moments (skewness).
\item Normal series approximation.
\end{itemize}
\end{frame}
\begin{frame}[fragile]
\frametitle{Pause 5.}
\begin{center}
Let us look into the code!
\end{center}
\end{frame}
\begin{frame}[fragile]
\frametitle{Normal approximation, critics.}
\begin{itemize}
\item What to do if one of the moments is infinite?
\item Is it OK to approximate a model with 3 parameters by 2 parameters?
\item We underestimate the right tail.
\end{itemize}
\end{frame}
\begin{frame}[fragile]
\frametitle{Test description.}
\begin{itemize}
\item frequency is Poisson, severity is Pareto.
\item mean frequency is from 0.1 to 100.
\item $\alpha$ from Pareto is from 2.5 to 10.
\item $x_m$ from Pareto is from 100000 to 100000000.
\item 90 test cases.
\end{itemize}
Criteria:
\begin{itemize}
	\item compare means
  \item compare VaR(0.95)
  \item compare performance
\end{itemize}
\end{frame}
\begin{frame}[fragile]
\frametitle{Results.}
\begin{figure}[h!]
\centering
\includegraphics[scale=0.5]{Results.jpg}
\end{figure}
\end{frame}
\begin{frame}[fragile]
\frametitle{What is missing?}
\begin{itemize}
	\item nonuniform discretizations
  \item algorithms for defining the default parameters
	\item convolution method is too tricky
\end{itemize}
\end{frame}
\begin{frame}[fragile]
\frametitle{What else is in \textbf{actuar}?}
\begin{itemize}
\item Many probability distributions.
\item Risk and ruin theory.
\item Credibility theory.
\item Support of mixture probability distributions.
\end{itemize}
\end{frame}
\end{document}