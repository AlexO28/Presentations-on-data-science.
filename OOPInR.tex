\documentclass{beamer}
\usepackage{listings}
\usetheme{Berlin}
\title{OOP in R}
\author{Alexey Osipov}
\date{11.02.2020}
\begin{document}
\maketitle
\begin{frame}[fragile]
\frametitle{Systems of classes in R.}
\begin{itemize}
\item S3-classes,
\item S4-classes,
\item RC- or R5-classes,
\item R6-classes.
\end{itemize}

\end{frame}
\begin{frame}[fragile]
\frametitle{Types of OOP.}
\begin{itemize}
	\item Message-passing OOP (Java, C++, C\#).
	
	A message is sent to an object, an object chooses a method.

 \begin{lstlisting}
       canvas.drawRect(�blue�)
   \end{lstlisting}

\item Generic-function OOP.

A generic function decides which method to call.
\end{itemize}

\end{frame}
\begin{frame}[fragile]
\frametitle{S3 classes}
\begin{itemize}
	\item Class is determined by class attribute.

\item Generic function: 
 \begin{lstlisting}
UseMethod(�mean�, x)
 \end{lstlisting}
We decide which mean we call basing on class:
 \begin{lstlisting}
mean.numeric, mean.data.frame,
 mean.matrix, mean.default.
 \end{lstlisting}

\item Inheritance: 
 \begin{lstlisting}
NextMethod.
 \end{lstlisting}
\end{itemize}
\end{frame}

\begin{frame}[fragile]
\frametitle{S4 classes}
\begin{itemize}
	\item Class has: name, representation (slots), contains (character vector of classes that it inherits from).

\item Create class with 
setClass,
 create instance with 
new,
 set method with 
setMethod.

\item In S4 we check the types of the slots.

\item We access a slot via @ or slot or [[.

\item S4 is stricter than S3, but still generic function OOP.


\item
 \begin{lstlisting}
setGeneric, setMethod.
 \end{lstlisting}

\end{itemize}
\end{frame}

\begin{frame}[fragile]
\frametitle{R5 or RC classes}
\begin{itemize}
	\item Message-passing OOP, mutability aka pass by reference.
	
	An implementation is environments + S4-classes.

\item Class is: contains, fields, methods.

\item It is possible to add methods after creation.

\item 
 \begin{lstlisting}
setRefClass, $new(), obj$method()
 \end{lstlisting}

We modify fields with 
 \begin{lstlisting}
<<-
 \end{lstlisting}
\end{itemize}
\end{frame}

\begin{frame}[fragile]
\frametitle{Drawbacks of R5 classes.}

\begin{enumerate}
	\item They are slow.
	\item They are not portable.
	\item No private methods or fields.
	\item 
We detect fields via
 \begin{lstlisting}
<<-
 \end{lstlisting}
 \begin{lstlisting}
bar <<- 1         #bar is a field
 \end{lstlisting}
 \begin{lstlisting}
bar <- 1           #bar is a variable
 \end{lstlisting}
\end{enumerate}

\end{frame}

\begin{frame}[fragile]
\frametitle{R6 classes}

\begin{itemize}
	\item Message-passing OOP, mutability aka pass by reference.
	\item public, private (aka protected)
	\item active binding (non-trivial getter)
	\item 
	 \begin{lstlisting}
	class$new(), obj$method(), obj$field.
 \end{lstlisting}
\item
we detect field via: 
	 \begin{lstlisting}
self$, private$
	 \end{lstlisting}
\item Inheritance.
\end{itemize}
\end{frame}

\begin{frame}[fragile]
\frametitle{R6 classes, continued}
\begin{itemize}
	\item Special treatment of fields passed by reference.
	\item Portable vs non-portable.
\item Possibility of class modification after creation.
\item Faster and simpler than R5-classes.
\item Debug is tricky:
	 \begin{lstlisting} 
class$debug('method')
	 \end{lstlisting} 
enables debug, then
	 \begin{lstlisting}
debug(object$method).
	 \end{lstlisting}
\end{itemize}
\end{frame}
\end{document}